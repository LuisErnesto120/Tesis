\documentclass[../main.tex]{subfiles}
\begin{document}
\chapter{Conclusiones}
La metodología descrita en el capítulo \ref{cap:metodologia} permite sintetizar ortoferritas \neod{} y \sama{} de alta pureza y caracterizar el acoplamiento entre sus propiedades estructurales, morfológicas, ópticas, eléctricas y magnéticas.

El análisis termogravimétrico revela un comportamiento complejo de las muestras al alcanzar temperaturas altas. Esto, junto con el desdoblamiento del pico principal que se observa a través de difracción de rayos X sugiere un cambio en la estructura cristalina a alta temperatura.

Se observa que la sonicación tuvo un efecto considerable en el porcentaje de partículas con diámetro $\leq1$ $\mu$m y el diámetro promedio de partícula, sin embargo este efecto no escala linealmente con el tiempo de sonicación, al duplicarlo se obtienen resultados similares. Se concluye que el proceso de sonicación rompe las partículas más grandes hasta un tamaño mínimo. Para continuar reduciendo el tamaño se requiere de un proceso que imparta una mayor cantidad de energía a las partículas.

Por otra parte, la espectroscopía UV-Vis reveló una dependencia del \textit{band gap} con la temperatura y el tiempo de sonicación.

Se observaron transiciones de fase magnéticas a baja temperatura, además de una separación de las curvas ZFC y FC, en particular en las muestras sonicadas, lo cual sugiere que esta técnica aumenta la anisotropía de las muestras. Esto puede deberse a la reducción del tamaño promedio de partícula.

Ambas muestras presentan comportamiento ferromagnético débil a bajas temperaturas, teniendo el \sama{} una respuesta mayor. Sin embargo, la única muestra que mantuvo este comportamiento a temperatura ambiente fue el \sama{}

Finalmente, las mediciones de polarización revelaron un comportamiento ferroeléctrico a bajo voltaje para ambas muestras, el cual se perdió al aumentar el voltaje máximo aplicado.

Considerando las mediciones magnéticas y eléctricas, se puede concluir que la muestra de \sama{} sonicada 4 h presenta multiferroicidad débil a temperatura ambiente. Sin embargo, es importante notar que el proceso de sinterización aumenta la temperatura de la muestra hasta 1000\gradoC{}, y debido al desdoblamiento del pico principal observado, el comportamiento de la muestra podría verse afectado por el proceso de sinterización.
\chapter{Perspectivas a futuro}
\begin{itemize}
    \item Medir las constantes de acoplamiento magnetoeléctrico de las muestras.
    \item Caracterizar magnética y eléctricamente las muestras calcinadas a temperaturas distintas.
    \item Buscar métodos de sinterización que no impliquen subir la temperatura de las muestras más allá de su temperatura de calcinación.
\end{itemize}
\end{document}