\documentclass[../main.tex]{subfiles}
\begin{document}
\chapter{Conclusiones}
Los TGA realizados muestran que la cristalización de ambas muestras comienza alrededor de los 300\gradoC{}, además se observa un cambio en el comportamiento térmico de ambas ortoferritas al aumentar la temperatura más allá de los 700\gradoC{}.

Mediante SEM se observaron partículas porosas y frágiles con una distribución heterogénea de tamaños en todas las temperaturas de calcinación. 

El proceso de sonicación permitió homogenizar el tamaño de partícula, lo cuál se refleja en un aumento en el porcentaje de partículas con diámetro $\leq1$ $\mu$m ($67.29\pm4.079$\%$\shortrightarrow87.10\pm1.292$\% para el \neod{}, $55.18\pm2.103$\%$\shortrightarrow92.62\pm0.915$\% para el \sama{}) y en una reducción del diámetro promedio ($1.49\pm0.028$ $\mu$m $\shortrightarrow0.98\pm0.023$ $\mu$m para el \neod{}, $2.01\pm0.051$ $\mu$m $\shortrightarrow1.66\pm0.045$ $\mu$m para el \sama{}).

Por su parte, se observó una disminución del tamaño de partícula con la temperatura de calcinación ($2.66\pm0.065$ $\mu$m $\shortrightarrow1.49\pm0.028$ $\mu$m para el \neod{}, $2.02\pm0.055$ $\mu$m $\shortrightarrow1.91\pm0.034$ $\mu$m para el \sama{}), sin embargo no existe una dependencia clara del porcentaje de partículas con diámetro $\leq1$ $\mu$m con la temperatura de calcinación.

La espectroscopía de dispersión de energía confirma la ausencia de contaminantes externos en las muestras, observando únicamente los elementos que componen a cada ortoferrita, además del carbono de la cinta utilizada para adherir las muestras al realizar las mediciones.

Haciendo una comparación cualitativa de los espectros obtenidos mediante DRX se puede observar un desdoblamiento del pico principal al aumentar la temperatura de calcinación, siendo este fenómeno más evidente en el \neod{}, esto, junto con los comportamiento de las ferritas a $T>700$\gradoC{}, sugiere un cambio en la estructura cristalina al variar la temperatura de calcinación. La sonicación no tuvo un efecto significativo en el patrón de difracción, por lo que no se observó un cambio en la estructura cristalina al realizar este proceso.

Por otro lado, los refinamientos Rietveld de estas muestras revelaron una pureza alta sin importar la temperatura de calcinación ni el tiempo de sonicación ($\geq95.73$\% para el \neod{} y $\geq97.56$ para el \sama{}), habiendo únicamente trazas de óxidos e hidróxidos en ambos materiales.

En conjunto, la caracterización térmica, estructural y morfológica revela que el proceso de síntesis descrito en el capítulo \ref{cap:metodologia} genera muestras de alta pureza, en forma de polvos compuestos de partículas porosas que se rompen fácilmente. Además, muestra una dependencia con la temperatura de calcinación no sólo del tamaño de partícula, sino de la estructura cristalina de las muestras.

La espectroscopía UV-Vis reveló que las muestras son semiconductores con \textit{band gaps} de alrededor de 2.2 eV, valor que depende tanto de la temperatura de calcinación como del tiempo de sonicación, ocurriendo el mínimo en las muestras sin sonicar calcinadas a menor temperatura.

En cuanto a las propiedades magnéticas, se observó un comportamiento superparamagnético en las muestras de \neod{} calcinadas a 600\gradoC{}, independientemente del tiempo de sonicación, a diferencia de la muestra calcinada a 900\gradoC{}, la cual muestra un comportamiento ferromagnético débil a $T=300 K$, consistente con el comportamiento de un antiferromagneto. Para estas muestras se observan transiciones de fase magnéticas a baja temperatura sólo en las muestras calcinadas a 600\gradoC{}.

Por su parte, las muestras de \sama{} no presentan un cambio tan drástico con la temperatura de calcinación, sin embargo sí presentan una mejoría en sus propiedades ferromagnéticas con la sonicación. Se observa un ciclo de histéresis para la muestra de \sama{} sonicada incluso a baja temperatura, contrario a la muestra sin sonicar. Para esta muestra ocurren reordenamientos magnéticos complejos a baja temperatura, los cuales se ven modificados por el tiempo de calcinación y de sonicación.

Finalmente, las mediciones de polarización revelaron una respuesta ferroeléctrica a bajo voltaje en ambas muestras, la cual desapareció al aumentar el voltaje máximo aplicado debido a que se trata de semiconductores. En el caso del \neod{}, las mediciones a 100 V tuvieron valores muy similares sin importar el tiempo de sonicación, mientras que la sonicación parece mejorar las propiedades ferroeléctricas de las muestras de \sama{}, aumentando su $E_c$ y $P_r$.

Se observa una dependencia de las propiedades ópticas, eléctricas y magnéticas de ambas ortoferritas con la temperatura de calcinación, sobretodo para las muestras de \neod{}, se piensa que esto se debe al cambio de estructura que sugiere el desdoblamiento del pico principal. Las propiedades de las muestras de \sama{} son más sensibles a la sonicación que las de las muestras de \neod{}, obteniendo mejorías tanto en sus propiedades ferromagnéticas como ferroeléctricas.

En conjunto, las mediciones de las propiedades magnéticas y eléctricas sugieren un ordenamiento de ambas propiedades, sin embargo, es importante mencionar que el proceso de sinterización aumenta la temperatura de la muestra hasta 1000\gradoC{}, lo cual podría afectar el comportamiento de la muestra debido al desdoblamiento del pico principal observado. Aún con esto, la presencia de ferromagnetismo débil en las muestras calcinadas a 900\gradoC{}, y el hecho de que no se observen cambios tan grandes en el patrón de difracción de rayos X entre las muestras calcinadas a 900 y 1000\gradoC{} sugieren que el \neod{} y \sama{} calcinados a $T\geq900$\gradoC{} presentan multiferroicidad, siendo necesario un proceso de sinterización a temperatura variable para confirmar esta propiedad.
\section{Perspectivas a futuro}
\begin{itemize}
    \item Medir las constantes de acoplamiento magnetoeléctrico de las muestras.
    \item Buscar métodos de sinterización que no impliquen subir la temperatura de las muestras más allá de su temperatura de calcinación para comparar propiedades eléctricas a distintas temperaturas de sinterización.
    \item Caracterizar magnética y eléctricamente las muestras calcinadas a temperaturas intermedias.
\end{itemize}
\end{document}