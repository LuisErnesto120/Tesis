\documentclass[../main.tex]{subfiles}
\begin{document}
\chapter{Conclusiones}
La metodología descrita en el capítulo \ref{cap:metodologia} permite sintetizar ortoferritas de \neod{} y \sama{} de alta pureza, además de caracterizar el acoplamiento entre sus propiedades estructurales, morfológicas, ópticas, eléctricas y magnéticas.

Los análisis termogravimétricos revelaron un comportamiento complejo de las muestras a altas temperaturas. Este fenómeno, junto con el desdoblamiento del pico principal observado a través de DRX, sugiere un cambio en la estructura cristalina a altas temperaturas.

Se observa que la sonicación mostró un efecto significativo en la distribución de tamaños de partícula, redujo el porcentaje de partículas con diámetro $\leq1$ $\mu$m y disminuyó el diámetro promedio. Sin embargo, este efecto no escaló linealmente con el tiempo de sonicación, ya que al duplicarlo se obtuvieron resultados similares. Esto indica que el proceso fragmenta las partículas más grandes hasta alcanzar un tamaño mínimo, y que una mayor reducción requeriría otras técnicas.

Por otro lado, los espectros UV-Vis mostraron que el \textit{band gap} depende tanto de la temperatura de calcinación como del tiempo de sonicación, ocurriendo el mínimo en las muestras sin sonicar calcinadas a menor temperatura.

En cuanto a las propiedades magnéticas, se observaron transiciones de fase a bajas temperaturas además de una gran separación entre las curvas ZFC y FC, particularmente en las muestras sonicadas, lo que sugiere que ésta incrementa el efecto de la anisotropía.

Ambas muestras presentan comportamiento ferromagnético débil a bajas temperaturas, teniendo el \sama{} una respuesta mayor. Sin embargo, la única muestra que mantuvo este comportamiento a temperatura ambiente fue el \sama{}

Finalmente, las mediciones de polarización revelaron una respuesta ferroeléctrica a bajo voltaje en ambas muestras, la cual desapareció al aumentar el voltaje máximo aplicado.

Considerando las mediciones magnéticas y eléctricas, se puede concluir que la muestra de \sama{} sonicada 4 h presenta multiferroicidad débil a temperatura ambiente. Sin embargo, es importante mencionar que el proceso de sinterización aumenta la temperatura de la muestra hasta 1000\gradoC{}, lo cual podría afectar el comportamiento de la muestra debido al desdoblamiento del pico principal observado.
\section{Perspectivas a futuro}
\begin{itemize}
    \item Medir las constantes de acoplamiento magnetoeléctrico de las muestras.
    \item Caracterizar magnética y eléctricamente las muestras calcinadas a temperaturas distintas.
    \item Buscar métodos de sinterización que no impliquen subir la temperatura de las muestras más allá de su temperatura de calcinación.
\end{itemize}
\end{document}