\documentclass[letterpaper]{article}
\usepackage[utf8]{inputenc}
\usepackage[spanish,es-tabla]{babel}
\usepackage{csquotes}
\usepackage{graphicx}
\usepackage{wrapfig}
\usepackage{verbatim}
\usepackage{amsmath, amssymb}
\usepackage{MnSymbol}
\usepackage{upgreek}
\usepackage{mathrsfs}
\usepackage[mathcal]{euscript}
\usepackage{physics}
\usepackage{multicol}
\usepackage[hidelinks]{hyperref}
\usepackage{endnotes}
\usepackage{array}
\usepackage{color}
\usepackage{lscape}
\usepackage{enumitem}
\usepackage{units}
\usepackage{gensymb}
\usepackage{glossaries}
\usepackage{import}
\usepackage{caption}
\usepackage{lipsum}
\usepackage{subfig}
\usepackage{float}
\usepackage{commath}
\usepackage[font=small]{caption}
\usepackage[left=2.5cm,right=2.5cm,top=2 cm,bottom=2 cm]{geometry}
\usepackage[backend=biber,sorting=none,style=numeric-comp,bibstyle=chem-acs,articletitle=true,biblabel=brackets,doi=false,autocite = superscript]{biblatex} %Nota del 05 de julio de 2021: esta librería genera las referencias bibliográficas.
\bibliography{bib}
\usepackage{verbatim} %Nota del 05 de julio de 2021: esta libreria es para comentar varias lineas
\usepackage{float} %Nota del 06 de julio de 2021: esta libreria sirve para las imagenes en el entorno multicols
\usepackage{ragged2e}
\usepackage[version=4]{mhchem}
\newcommand{\e}{\hat{\textbf{e}}} %para los vectores unitarios
\newcommand{\im}{\mathrm{i}} %para las partes imaginarias, es la i
\newcommand{\neod}{\ce{NdFeO3}}
\newcommand{\sama}{\ce{SmFeO5}}
\newcommand{\grado}{$^\circ$}
\newcommand{\gradoC}{$^\circ$C}


\begin{document}

\includegraphics[width=3cm]{IF.jpg}
\vspace{-4cm}

\begin{flushleft}
\hspace{3.85 cm}\LARGE \textbf{Reporte Mensual para Servicio Social 3.}\\
\hspace{3.75 cm} Medición y análisis de magnetización contra\\
\hspace{3.75 cm} campo magnético y contra Temperatura mediante\\
\hspace{3.75 cm} magnetometría SQUID\\
\end{flushleft}


\begin{flushleft}
\hspace{3.75 cm} \large Servicio Social: Investigación Instituto de Física (2023-12/82-2509).\\
\hspace{3.75 cm} Espintrónica: dinámica de momentos de espín.\\
\hspace{3.75 cm} Luis Ernesto Garduño Garrido\\
%\hspace{3.75 cm} en caso de ser necesario una segunda línea
%\hspace{3.75 cm} \textbf{} En caso de ser necesaria otra línea
\vspace{0.25 cm}
\normalsize \hspace{3.75 cm} e-mail: luiseg@estudiantes.fisica.unam.mx \\
\hspace{3.75 cm} Entregado: /Junio/2024
\end{flushleft}
\hrule
\begin{minipage}[t]{4.5 cm}
\vspace{5 pt}
\centering \footnotesize\textbf{Palabras Clave}
\justify
. \neod\\
. \\
. \\
. \\
. \\
. \\
. 
\end{minipage}
\begin{minipage}[t]{12 cm}
\vspace{5 pt}
\begin{abstract}
    hola
\end{abstract}
\end{minipage}
\vspace{5 pt}
\hrule
\section{Test}
a
\end{document}